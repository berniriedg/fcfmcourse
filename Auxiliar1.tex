\documentclass{dcccourse}
\usepackage{introprogra}
\usepackage[utf8]{inputenc}


\num{1}
\title{Autómatas finitos y no deterministas}
\course{CC1002 - Introducción a la Programación}
\professor{Alejandro Hevia}
%\professor[gnavarro@dcc.uchile.cl]{Gonzalo Navarro}
\assistant[nlehmann@dcc.uchile.cl]{Nicolás Lehmann}
\assistant{José Garrido}


\begin{document}

\problem
La expansión en serie de Taylor de la función $\cos(x)$ en torno al cero es:
$$\cos (x) = \sum^{\infty}_{n=0} \frac{(-1)^n}{(2n)!} x^{2n}$$

Considerando hasta el cuarto término podemos aproximar la función de la siguiente forma:
$$\cos  (x) \approx 1-\frac{x^2}{2!}+\frac{x^4}{4!}-\frac{x^6}{6!} $$

Escriba una expresión en python que calcule aproximadamente el valor $\cos (0.785)$. ?`Qué harías si tienes que calcular la función en más de un punto del dominio?

\problem
Considere los programas listados a continuación. ?`Qué imprime el interprete en cada caso?

\begin{minted}[frame=lines,bgcolor=LightGray]{python}
  >>> a = 1 
  >>> b = 2
  >>> a b +
\end{minted}
\begin{minted}[frame=lines,bgcolor=LightGray]{python}
  >>> a = "Hola"
  >>> a * "tarola"
\end{minted}
\begin{minted}[frame=lines,bgcolor=LightGray]{python}
  >>> def sumToN(n):
  ...        a = n*(n+1)
  ...    return a/2
\end{minted}
\begin{minted}[frame=lines,bgcolor=LightGray]{python}
  >>> def foo(x):
  ...    return x/2 - 1
  >>> a = foo(3)
  >>> 8/a
\end{minted}

\problem
Utilizando la ``Receta de Diseño'' vista en clases. Defina la función \verb|pesosADolar| que recibe un valor numérico correspondiente a la cantidad que se tiene en Pesos y que se desean transformar en Dolares. Considere que: 
1 Dolar = 571 Pesos

Modifique la función \verb|pesosADolar| de manera de que dicha función reciba 2 argumentos, el primero de ellos corresponde a la cantidad en Pesos que se desea transformar, mientras que el segundo argumento corresponde al factor de cambio que se utilizará.

?`Cómo podría generar una función que generalice el cambio de una divisa por otra entregando la cantidad y el factor de cambio?

\problem
Considere los programas listados a continuación. ?`Qué imprime el interprete en cada caso? Explique la razón del comportamiento obtenido en cada caso.

\begin{minted}[frame=lines,bgcolor=LightGray]{python}
  >>> densidad_cobre = 8960
  >>> def volumen_cobre(masa):
  ...     return masa/densidad_cobre
  >>> volumen_cobre(2500)
  >>> densidad_cobre = 8960.0
  >>> volumen_cobre(2500)
\end{minted}
\begin{minted}[frame=lines,bgcolor=LightGray]{python}
  >>> g = 9.78
  >>> def peso(masa):
  ...     g = 10
  ...     return masa*g
  >>> peso(50)
  >>> g
\end{minted}
\begin{minted}[frame=lines,bgcolor=LightGray]{python}
  >>> def presion(fuerza):
  ...     area = 100
  ...     return fuerza/area
  >>> presion(1000)
  >>> area

\end{minted}

\end{document}
