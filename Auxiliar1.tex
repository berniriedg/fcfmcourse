\documentclass[dcc]{fcfmcourse}
\usepackage{teoria}
\usepackage[utf8]{inputenc}

\title[1]{Autómatas finitos y no deterministas}
\course[CC3102]{Teoría de la Computación}
\professor[jperez@dcc.uchile.cl]{Jorge Pérez}
\assistant[nlehmann@dcc.uchile.cl]{Nicolás Lehman}
\assistant[ralonso@dcc.uchile.cl]{Rodrigo Alonso}
% Si pasas el comando usedate a la clase, la fecha aparecerá bajo la lista de auxiliares.
% Puedes usar el formato de fecha por defecto de latex (y traducirla usando babel)
% o puedes escribir lo que quieras con el comando \date.
% \date{1 de Septiembre, 2015}


\begin{document}
\maketitle

\begin{problems}
\problem Dando un diagrama de estados construya un AFD que acepte cada uno de los siguientes lenguajes.

\begin{enumerate}
\item \lang{$L_1$}{$w\in \set{a,b}^*$}{$w$ tiene un numero impar de $a$'s y un número par de $b$'s}
  \begin{solution}
    \begin{VCPicture}{(-1,-1)(4,4)}
    \State[q_0]{(0,0)}{0}
    \State[q_1]{(3,0)}{1}
    \FinalState[q_2]{(0,3)}{2}
    \State[q_3]{(3,3)}{3}
    \Initial{0}

    \ArcR{0}{1}{a}
    \ArcR{0}{2}{b}
    \ArcR{1}{0}{a}
    \ArcR{1}{3}{b}
    \ArcR{2}{0}{b}
    \ArcR{2}{3}{a}
    \ArcR{3}{2}{a}
    \ArcR{3}{1}{b}
    \end{VCPicture}
  \end{solution}
\item \lang{$L_2$}{$w\in \set{0,1}^*$}{$w$ no tiene ni $00$ ni $11$ como subcadena}
\end{enumerate}

\problem Sea $\Sigma$ un alfabeto tal que $|\Sigma|=n$. Construya un AFD que acepte el lenguaje de los strings $w$ en $\Sigma^*$ donde cada símbolo aparece al menos dos veces. Por ejemplo, si $\Sigma=\set{a,b,c,d}$ el string $abbcbacdd$ pertenece al lenguaje, pero $abdbccbd$ y $aabbcc$ no.

\begin{solution}
  Representaremos los estados con $n$-tuplas donde cada posición corresponde a cuantas veces ha sido visto cada símbolo hasta el momento.
  Como sólo es necesario saber si cada símbolo aparece al menos dos veces, paramos de contar en el 2.
  Sin perdida de generalidad asumimos que $\Sigma = \set{1,2,\ldots,n}$.

  Defimos entonces el AFD $(Q, \Sigma, \delta, q_0, F)$ tal que:
  \begin{itemize}
    \item $Q = \set{0,1,2}^n$
    \item $q_0 = \vec{0}$
    \item $F = \set{\vec{2}}$
    \item $\delta(\langle a_1, \ldots, a_i,\ldots a_n \rangle, i) =
           \langle a_1,\ldots, a_i \oplus_2 1, \ldots a_n \rangle$
  \end{itemize}
  Donde 
  \[a \oplus_n b = 
   \begin{cases}
     a + b & a + b \leq n \\
     n & a + b > n
   \end{cases}
  \]
\end{solution}

\problem Construya un diagrama de estados para un AFND que acepte cada uno de los siguiente lenguajes.

\begin{enumerate}
\item \lang{$L_3$}{$w\in \set{a,b}^*$}{cada $a$ en $w$ esta antecedida y sucedida por una $b$}
\item \lang{$L_4$}{$w\in \set{b,e,r}^*$}{$w$ contiene $beer$ como subcadena}
\end{enumerate}

\problem Sea $\Sigma$ un alfabeto de tamaño $n$. Construya un AFND que acepte el siguiente lenguaje sobre $\Sigma\cup \{\#\}$

\begin{center}
\lang{$L_5$}{$\expand{w}{k}\#w_{k+1}$}{$w_i \in \Sigma$ y $\exists i\in\ords{k}, w_i=w_{k+1}$}
\end{center}

?`Qué tamaño tiene el AFND en función de $n$? ?`Qué puede decir del tamaño de un AFD que acepta el mismo lenguaje?

\end{problems}
\end{document}
